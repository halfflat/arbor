\documentclass[parskip=half]{scrartcl}

\usepackage{amsmath}
\usepackage{amssymb}
\usepackage[citestyle=authoryear-icomp,bibstyle=authoryear]{biblatex}
\usepackage[style=british]{csquotes}
\usepackage{fontspec}

\setmainfont[Numbers=Lowercase]{TeX Gyre Pagella}
\newfontfamily\dispfamily{TeX Gyre Pagella}
\addtokomafont{disposition}{\dispfamily}

\MakeAutoQuote{«}{»}

\addbibresource{stability.bib}

\title{Eigenvectors for the discrete cable Laplacian}
\author{Sam Yates}
\date{October 20, 2016}

\begin{document}

\maketitle

\section{Background}

When the cable equation is uniformly discretized with vertex-centred FVM and
zero-flux boundary conditions, the Laplacian operator takes the form
\[
    h\cdot L_{i, j} =
    \begin{cases}
	1 & \text{if $i=j=1$ or $i=j=n$,}\\
	2 & \text{if $1<i=j<n$,}\\
	-1 & \text{if $|i-j|=1$,}\\
	0 & \text{otherwise,}
    \end{cases}
\]
where there are $n$ vertices on the cable of length $L$, and the spatial
step size $h=L/(n-1)$.

What are the eigenvalues and eigenvectors of $L$?

\section{Solution}

Let the eigenvalues be $\lambda_j$ for $j=1,\dots,n$, and $e^{(j)}$
be a non-zero eigenvector in the corresponding eigenspace.

From $L e^{(j)}=\lambda_j e^{(j)}$ we have the relations on the components
of the eigenvectors as follows:
\begin{align}
    \label{eq:evec1}
    e^{(j)}_2 &= (1-\lambda_j) e^{(j)}_1,\\
    \label{eq:evec2}
    e^{(j)}_{k+2} &= (2-\lambda_j) e^{(j)}_{k+1} - e^{(j)}_k \quad\text{for $k=0,\dots,n-2$},\\
    \label{eq:evec3}
    e^{(j)}_{n-1} &= (1-\lambda_j) e^{(j)}_n.
\end{align}

Let $E_k(x)$ be the $k$-degree polynomial in $x$ such that
$e^{(j)}_k = E_{k-1}(1-\lambda_j/2) e^{(j)}_1$, so that the recurrence relation given by
(\ref{eq:evec1}) and (\ref{eq:evec2}) becomes
\begin{align}
    E_0(x) &= 1,\\
    E_1(x) &= 2x-1,\\
    \label{eq:erec}
    E_{k+2}(x) &= 2x E_{k+1}(x) - E_k(x).
\end{align}

The recurrence relation is the same as that for the Chebyshev polynomials of
the second kind $U_k(x)$, which satisfy
\begin{align*}
    U_0(x) &= 1,\\
    U_1(x) &= 2x,\\
    U_{k+2}(x) &= 2x U_{k+1}(x) - U_k(x),
\end{align*}
and which have roots
\[
    U_{n-1}(x) = 0 \implies x = \cos \frac{j\pi}{n}, \quad\text{j=1,\dots,n-1}.
\]

$E_k(x)$ then can be expressed as a linear combination (in the ring
of polynomials) of the $U_k(x)$. Allowing $U_{-1}(x)=0$,
\begin{equation}
    \label{eq:easu}
    E_k(x) = U_k(x)-U_{k-1}(x).
\end{equation}
(This is in fact a special case of \cite[Theorem 4.1]{aharonov2005}.)

Solutions $\lambda_j$ to (\ref{eq:evec3}) then are of the form $2-2x$
for $x$ satisfying
\[
    (2x-1)E_{n-1}(x)=E_{n-2}(x).
\]
Applying (\ref{eq:easu}) and the recurrence relationship gives
\[
    U_{n-1}(x)(2x-2) = 0,
\]
with solutions $x_j = \cos \frac{j\pi}{n}$ for $j=1,\dots,n-1$ and
$x_n = 0$, giving the eigenvalues
\[
    \lambda_j = 2 - 2 \cos \frac{j\pi}{n}, \quad j=1,\dots, n.
\]

The $U_k$ satisfy the trigonometric identity
\[
    U_k(\cos \alpha) = \frac{\sin (k+1)\alpha}{\sin \alpha},
\]
so
\[
    E_{k-1}(1-\frac{\lambda_j}{2})
    = \frac{\sin \frac{jk\pi}{n} - \sin \frac{j(k-1)\pi}{n}}{\sin \frac{j\pi}{n}}.
\]
Applying the sum-to-product trigonometric identity then gives
\[
    \cos \frac{j\pi}{2n} E_{k-1}(1-\frac{\lambda_j}{2}) =
    \cos \frac{j\pi}{n}(k-\frac{1}{2}).
\]
Without loss of generality, we can take the first element of the
$j$th eigenvector $e^{(j)}_1$ to be $\cos j\pi/2n$, so that
\[
    e^{(j)}_k = \cos \frac{j\pi}{n}(k-\frac{1}{2}),\quad\text{for $j,k=1,\dots,n$}.
\]

\printbibliography
\end{document}
