\documentclass[parskip=half]{scrartcl}

\usepackage{amsmath}
\usepackage{amssymb}
\usepackage[citestyle=authoryear-icomp,bibstyle=authoryear]{biblatex}
\usepackage{booktabs}
\usepackage[style=british]{csquotes}
\usepackage{fontspec}
\usepackage{mathtools}
\usepackage{siunitx}
\usepackage{tikz}
\usetikzlibrary{angles,calc,intersections,quotes}

\setmainfont[Numbers=Lowercase]{TeX Gyre Pagella}
\newfontfamily\dispfamily{TeX Gyre Pagella}
\addtokomafont{disposition}{\dispfamily}

\MakeAutoQuote{«}{»}
\DeclareMathOperator{\Res}{Res}

\addbibresource{cable.bib}

\newcommand{\Int}[2]{\int_{#1}^{#2}\!}
\newcommand{\D}{\mathop{}\!d}

\title{Analytic solutions to the cable equation}

\author{Sam Yates}
\date{July 29, 2019}

\begin{document}

\maketitle

\section{The cable equation}

The cable equation describes the evolution of the potential
on a long, thin, conducting cable in a conducting medium, separated
by a leaky dielectric. It assumes that the behaviour
can be modelled entirely as a one dimensional problem,
it terms of the linear conductivity of the cable $\sigma$,
and the conductance $g$ and capacitance $c$ per unit length of the
dielectric. The potential $v(x, t)$ then satisfies
\begin{equation}
    (\sigma v')' = c \dot v + g v.
\end{equation}
where $f'$ denotes the derivative of $f$ with respect to the first
variable, and $\dot{f}$ the derivative with respect to the second.
The quantities $\sigma$, $g$, and $c$ are functions of position,
and will depend on the electrical and geometrical properties of
the system.

For a cable of radius $r(x)$ with constant bulk resistivity
$R_L$, areal capacitance $C_M$ and areal surface resistivity
$R_M$, these parameters are given by
\begin{align}
    \sigma(x) &= \pi r(x)^2 / R_L, \\
    g(x) &= 2 \pi r(x) \sqrt{1 + r'(x)^2} / R_M, \\
    c(x) &= 2 \pi r(x) \sqrt{1 + r'(x)^2} C_M.
\end{align}
Letting $\theta(x) = \arctan r'(x)$, the cable equation becomes
\begin{equation}
    \label{eq:constelec}
    \frac{R_M}{2 R_L}\left(
	2\sin\theta\cdot v' + r \cos\theta\cdot v''
    \right) =
    R_M C_M \dot v + v.
\end{equation}

\subsection{Approximating the gradient}

In a finite volume discretization there
will be a computation of the approximation to the gradient
$v'(x)$ as a linear combination of two (or potentially more)
values of the discrete approximation of the voltage for points
near $x$.

If the coefficients in this approximation are constant in
time, they cannot account for source terms or values of the
voltage outside the points in question. The most faithful
such approximation should then reproduce the exact gradient $v'(x)$
in the source-free steady state form of the cable equation,
\begin{equation}
    (\sigma v')' = 0.
\end{equation}

The voltage $v$ is then determined by its values at
any two distinct points $a$ and $b$,
\begin{equation}
    \frac{v(x) - v(a)}{v(b) - v(a)} =
    \frac{\displaystyle \Int{a}{x} \sigma(z)^{-1} \D z}
	 {\displaystyle \Int{a}{b} \sigma(z)^{-1} \D z},
\end{equation}
and correspondingly, the gradient is
\begin{equation}
    \sigma(x) v'(x) =
    \frac{v(b) - v(a)}
	 {\displaystyle\Int{a}{b} \sigma(z)^{-1} \D z}.
\end{equation}
If $x$ is a point of discontinuity in $\sigma$,
the flux $\sigma(x)v'(x)$ is nonetheless well defined.

For a finite volume approximation, however, the available
estimates may correspond instead to surface area-weighted means
over control volumes. Consider two adjacent control volumes
on $[a, m]$ and $[m, b]$, with mean voltages $\bar v_a$ and
$\bar v_b$ given in terms of a weight function $w(x)$:
\begin{align}
    \bar v_a &= w_a^{-1} \Int{a}{m} w(x)v(x) \D x,\\
    \bar v_b &= w_b^{-1} \Int{m}{b} w(x)v(x) \D x,
\end{align}
with normalizing constants $w_a$ and $w_b$. With $\sigma v' = \kappa$
constant,
\begin{equation}
    \begin{aligned}
	\bar v_a
	&= w_a^{-1} \Int{a}{m} w(x) \left( v(m) + \kappa \Int{m}{x} \sigma(y)^{-1} \D y \right) \D x \\
	&= v(m) + \kappa \cdot w_a^{-1} \Int{a}{m} w(x) \Int{m}{x} \sigma(y)^{-1} \D y \D x,
    \end{aligned}
\end{equation}
and similarly, 
\begin{equation}
    \begin{aligned}
	\bar v_b
	&= v(m) + \kappa \cdot w_b^{-1} \Int{m}{b} w(x) \Int{m}{x} \sigma(y)^{-1} \D y \D x,
    \end{aligned}
\end{equation}
Taking the difference then gives a solution for $\kappa$,
\begin{equation}
    \kappa = (\bar v_b - \bar v_a) \cdot \left[
	w_b^{-1}\!\Int{m}{b} w(x)\! \Int{m}{x} \sigma(y)^{-1} \D y \D x\,-\,
	w_a^{-1}\!\Int{a}{m} w(x)\! \Int{m}{x} \sigma(y)^{-1} \D y \D x
	\right]^{\mathrlap{-1}}.
\end{equation}

\section{Analytic solutions for $v(x, t)$}

Two important special cases for the cable equation are the cylinder,
with $r(x)$ constant and $\theta(x)=0$, and the conical frustrum,
with $\theta(x)$ constant and $r(x)=x\tan\theta$.

Given constant electrical properties, as in (\ref{eq:constelec}),
a zero voltage at $t=0$, and Neumann boundary conditions, what
is $v(x, t)$?

\subsection{The cylinder}

Suppose the cable lies on the interval $[0, B]$, with boundary
conditions $v'(0, t) = 0$ and $v'(B, t) = I$ and initial value
$v(x, 0) = 0$.

A change of variables gives
\begin{equation}
    v(x, t) = \lambda I\cdot u(x/\lambda, t/\tau),
\end{equation}
where
\begin{equation}
    \lambda = \sqrt{\frac{R_M r}{2 R_L}}, \quad \tau = R_M C_M, \quad b = B/\lambda,
\end{equation}
and $u$ satisfies
\begin{equation}
    \begin{aligned}
	u''(x, t) &= \dot u(x, t) + u(x, t),\\
	u(x, 0) &= 0,\\
	u'(0, t) &= 0,\\
	u'(b, t) &= 1.
    \end{aligned}
\end{equation}

The corresponding time-invariant problem has solution $\hat u(x)$ where
\begin{equation}
    \hat u'' = \hat u,\quad \hat u'(0) = 0,\quad \hat u'(b) = 1,
\end{equation}
giving
\begin{equation}
    \hat u(x) = \frac{cosh x}{sinh b}.
\end{equation}

\begin{enumerate}
    \item \textbf{TODO:} dynamic solution via S-L.
\end{enumerate}

\subsection{The tapered cable}

Consider the cable as a conical frustrum on the interval $[A, B]$ with radius
$r(x) = x \tan\theta$ and boundary conditions $v'(A, t) = 0$, $v'(B, t) = I$,
and $v(x, 0) = 0$ (see Figure \ref{fig:tapered}). Reparameterizing gives
\begin{equation}
    v(x, t) = \lambda I\cdot u(x/\lambda, t/\tau),
\end{equation}
where
\begin{equation}
    \lambda = \frac{R_M\sin\theta}{2 R_L}, \quad \tau = R_M C_M, \quad a = A/\lambda, \quad b = B/\lambda,
\end{equation}
and $u$ satisfies
\begin{equation}
    \begin{aligned}
	x u''(x, t) + 2 u'(x, t) &= \dot u(x, t) + u(x, t),\\
	u(x, 0) &= 0,\\
	u'(a, t) &= 0,\\
	u'(b, t) &= 1.
    \end{aligned}
\end{equation}


%\section{Model}
%
%The cable equation is given by
%\[
%    S'(x)c(x)\frac{\partial v(x,t)}{\partial t} =
%    \frac{\partial}{\partial x}\left(\sigma(x)\frac{\partial v(x,t)}{\partial x}\right) -
%    S'(x)\frac{1}{\rho(x)}v(x,t),
%\]
%where $\sigma(x)$ is the axial conductivity, $\rho(x)$ is the surface areal
%resistivity, $c(x)$ is the surface areal capacitance, and $S(x)$ is the surface
%area up to the point $x$.
%
%Consider a tapered cable as per Figure~\ref{fig:tapered}, with constant bulk resistivity
%$\rho_V$, surface resistivity $\rho_M$, and areal capacitance $c$, so that
%\begin{align*}
%    S'(x) &= 2\pi x\tan\theta \sqrt{1+\tan^2\theta} = 2\pi x \tan\theta\csc\theta, \\
%    \sigma(x) &= \frac{1}{\rho_V} \pi (x\tan\theta)^2,
%\end{align*}
%and the cable equation becomes
%\[
%   1.
%\]

\begin{figure}[bh]
    \begin{tikzpicture}[scale=0.8]
        \coordinate (origin) at (0,0);
        \coordinate (b0) at (15,0);
        \coordinate (b1) at ($ (b0) + (0,1.8) $);
        \coordinate (a0) at (5,0);

        \path [name path=tangent] (origin) -- (b1);
        \path [name path=avert] (a0) -- +(0,10);
        \path [name intersections={of=tangent and avert, by=a1}];

        \coordinate (a-1) at ($ (a1)!2!(a0) $);
        \coordinate (b-1) at ($ (b1)!2!(b0) $);

        \draw [name path=axis, dashed] (origin) -- (b0);

        \draw let \p1 = (a1), \n2 = {0.2*\y1} in
            (a1) arc (90:270:\n2 and \y1);

        \draw let \p1 = (a1), \n2 = {0.2*\y1} in
            [densely dotted] (a-1) arc (-90:90:\n2 and \y1);

        \draw let \p1 = (b1), \n2 = {0.2*\y1} in
            (b0) ellipse (\n2 and \y1);

        \draw (a1) -- (b1) (a-1) -- (b-1);
        \draw [dotted] (origin) -- (a1) (origin) -- (a-1);

        \pic [draw, "$\theta$", angle radius = 15ex, angle eccentricity=1.2] {angle = b0--origin--b1};

        \coordinate (lright) at ($ (b1)!2!(b0) - (0,40pt) $);
        \coordinate (lleft) at ($ (origin) + (lright) - (b0) $);

        \path (lleft) node (x0label) { $x=0$ };
        \path (lright) node (xblabel) { $x=B =b\lambda $ };
        \path ($ (lleft)!(a0)!(lright) $) node (xalabel) { $x=A =a\lambda$ };

        \draw [dotted] ($ (a-1)-(0,2ex) $) -- ($ (xalabel)+(0,2ex) $);
        \draw [dotted] ($ (b-1)-(0,2ex) $) -- ($ (xblabel)+(0,2ex) $);
        \draw [dotted] ($ (origin)-(0,2ex) $) -- ($ (x0label)+(0,2ex) $);
    \end{tikzpicture}
    \caption{Tapered cable coordinates.}
    \label{fig:tapered}
\end{figure}

%\printbibliography
\end{document}
